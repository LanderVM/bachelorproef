%==============================================================================
% Sjabloon onderzoeksvoorstel bachproef
%==============================================================================
% Gebaseerd op document class `hogent-article'
% zie <https://github.com/HoGentTIN/latex-hogent-article>

\documentclass{hogent-article}

% Invoegen bibliografiebestand
\addbibresource{voorstel.bib}

\studyprogramme{Professionele bachelor toegepaste informatica}
\course{Bachelorproef}
\assignmenttype{Onderzoeksvoorstel}

\academicyear{2023-2024}

\title{Ontwikkeling van een webapplicatie voor het beoordelen van computervaardigheden: een strategie om de digitale kloof op de werkvloer te verkleinen.}

\author{Lander {Van Molle}}
\email{lander.vanmolle@student.hogent.be}


\supervisor[Co-promotor]{J. {Van der Poten} (Inetum-Realdolmen, \href{mailto:jordi.vanderpoten@inetum-realdolmen.world}{jordi.vanderpoten@inetum-realdolmen.world})}

\specialisation{Mobile \& Enterprise development}
\keywords{webapplicatie, digitale kloof, computervaardigheden testen}

\begin{document}

\begin{abstract}
  %Hier schrijf je de samenvatting van je voorstel, als een doorlopende tekst van één paragraaf. Let op: dit is geen inleiding, maar een samenvattende tekst van heel je voorstel met inleiding (voorstelling, kaderen thema), probleemstelling en centrale onderzoeksvraag, onderzoeksdoelstelling (wat zie je als het concrete resultaat van je bachelorproef?), voorgestelde methodologie, verwachte resultaten en meerwaarde van dit onderzoek (wat heeft de doelgroep aan het resultaat?).
  %Enkel de methodologie van de abstract moet in de verleden tijd staan
  In de huidige samenleving zijn computervaardigheden van groot belang. Door de toenemende digitalisering van werkprocessen in het dagelijks leven worden individuen met een lager opleidingsniveau, lager inkomen of hogere leeftijd vaak geconfronteerd met digitale uitsluiting vanwege een gebrek aan basiscomputervaardigheden. Dit leidt tot een aanzienlijke digitale kloof, vooral merkbaar op de werkvloer. Om deze uitdaging aan te pakken, stelt dit onderzoek voor om een webapplicatie te ontwikkelen die de digitale vaardigheden van gebruikers op de werkvloer beoordeelt. Naargelang het resultaat van de vaardigheidstesten biedt de werkgever de toekomstige of huidige werknemer de mogelijkheid om ICT-cursussen te volgen. Om dit te bereiken werden de meest cruciale vaardigheden geselecteerd en werden er testen ontwikkeld om de ICT-competenties van de gebruikers te beoordelen. De tests werden uitgevoerd binnen een Windows-omgeving waarbij de focus lag op scenario's die de vaardigheden in informatieverwerking, communicatie, bestandsbeheer (zoals het lokaliseren van bestanden op de pc en software-installatie) en tekstverwerking toetsten. Na afloop van de testen kreeg de gebruiker een overzicht van zijn prestaties zodat hij eventuele tekortkomingen kan verbeteren door het volgen van specifieke ICT-cursussen. De tool is een mogelijke oplossing om de digitale kloof te verkleinen aangezien individuen de mogelijkheid krijgen om zich bij te scholen. In de bedrijfssector kunnen werkgevers sollicitanten en personeel testen op eventuele tekortkomingen op het vlak van ICT en hen workshops of cursussen aanbieden. Dit maakt de tool waardevol voor zowel individuen als bedrijven.
\end{abstract}

\tableofcontents

%---------- Inleiding ---------------------------------------------------------

\section{Introductie}%
\label{sec:introductie}

%Waarover zal je bachelorproef gaan? Introduceer het thema en zorg dat volgende zaken zeker duidelijk aanwezig zijn:

%\begin{itemize}
%  \item kaderen thema
%  \item de doelgroep
%  \item de probleemstelling en (centrale) onderzoeksvraag
%  \item de onderzoeksdoelstelling
%\end{itemize}

%Denk er aan: een typische bachelorproef is \textit{toegepast onderzoek}, wat betekent dat je start vanuit een concrete probleemsituatie in bedrijfscontext, een \textbf{casus}. Het is belangrijk om je onderwerp goed af te bakenen: je gaat voor die \textit{ene specifieke probleemsituatie} op zoek naar een goede oplossing, op basis van de huidige kennis in het vakgebied.

%De doelgroep moet ook concreet en duidelijk zijn, dus geen algemene of vaag gedefinieerde groepen zoals \emph{bedrijven}, \emph{developers}, \emph{Vlamingen}, enz. Je richt je in elk geval op it-professionals, een bachelorproef is geen populariserende tekst. Eén specifiek bedrijf (die te maken hebben met een concrete probleemsituatie) is dus beter dan \emph{bedrijven} in het algemeen.

%Formuleer duidelijk de onderzoeksvraag! De begeleiders lezen nog steeds te veel voorstellen waarin we geen onderzoeksvraag terugvinden.

%Schrijf ook iets over de doelstelling. Wat zie je als het concrete eindresultaat van je onderzoek, naast de uitgeschreven scriptie? Is het een proof-of-concept, een rapport met aanbevelingen, \ldots Met welk eindresultaat kan je je bachelorproef als een succes beschouwen?

%introductie tot het onderwerp
%Een correct registratie van de gepresteerde uren is voor veel bedrijven, waaronder Inetum-Realdolmen, een zeer belangrijke kwestie. Deze uren zijn de basis om facturatie te kunnen doen en op deze manier inkomsten te vergaren uit geleverde diensten. Onder deze tijdsregistratie verstaan we het proces van het registreren, ook wel bijhouden, van uren die werknemers aan een bepaalde taak besteden.
In het huidige bedrijfsklimaat, waar flexibel werken en mobiele toegankelijkheid steeds meer de norm worden, is de noodzaak voor geavanceerde tijdregistratiesystemen belangrijker dan ooit. Accurate tijdregistratie speelt een cruciale rol in diverse aspecten van bedrijfsvoering, waaronder projectmanagement, loonadministratie, en klantfacturatie. Volgens een artikel door \textcite{Gupta2023} over tijdregistratie in projectmanagement, kan de implementatie van dergelijke systemen helpen om de productiviteit te verhogen, overuren te verminderen en projectbudgetten nauwkeuriger te beheren. Inetum-Realdolmen, een IT-consultancybedrijf, staat voor de uitdaging om deze processen te optimaliseren en te automatiseren om de efficiëntie en nauwkeurigheid te verhogen. Bovendien benadrukt de potentie van real-time tracking en tracing systemen binnen de logistieke netwerken, wat aangeeft hoe essentieel accurate tijdregistratie is voor het verbeteren van de operationele efficiëntie en het leveren van waarde aan klanten \autocite{TrackingAndTracingSystems}.

Dit onderzoek richt zich op de ontwikkeling van een tijdsregistratie-app voor Android, die gebruikmaakt van locatiegebaseerde technologieën om een geautomatiseerde registratie van werkuren mogelijk te maken. Dit systeem omvat niet alleen de nauwkeurigheid van tijdregistratie te verbeteren, maar ook administratieve lasten te verminderen en inzicht te verschaffen in de werkpatronen van medewerkers.

Centraal in dit onderzoek staat de vraag 'Hoe kan een tijdsregistratie-app bijdragen tot het verifiëren en rapporteren van werkuren met als doel correcte facturatie en opvolging?'. Hierbij zullen de volgende deelvragen worden onderzocht: Welke functionaliteiten moet de Android-app bevatten om gemakkelijk te kunnen worden gebruikt door zowel werknemers als management? Welke technologieën zijn het meest geschikt voor het nauwkeurig vaststellen van de locatie van een gebruiker om het registratieproces te vereenvoudigen? En hoe kan de verzamelde data op een veilige manier naar een centrale database worden verstuurd?

De kern van dit onderzoek richt zich op het ontwerpen van een Android app met bijhorende Java Enterprise Edition back-end die niet alleen waardevol is voor het management en de HR-afdeling binnen Inetum-Realdolmen, maar ook voor de werknemers zelf door het bieden van een transparante en eerlijke registratie van hun arbeidstijd. Het uiteindelijke resultaat van deze studie is een proof-of-concept voor een tijdsregistratie-app.

%---------- Stand van zaken ---------------------------------------------------

\section{Literatuurstudie}%
\label{sec:state-of-the-art}

%Hier beschrijf je de \emph{state-of-the-art} rondom je gekozen onderzoeksdomein, d.w.z.\ een inleidende, doorlopende tekst over het onderzoeksdomein van je bachelorproef. Je steunt daarbij heel sterk op de professionele \emph{vakliteratuur}, en niet zozeer op populariserende teksten voor een breed publiek. Wat is de huidige stand van zaken in dit domein, en wat zijn nog eventuele open vragen (die misschien de aanleiding waren tot je onderzoeksvraag!)?

%Je mag de titel van deze sectie ook aanpassen (literatuurstudie, stand van zaken, enz.). Zijn er al gelijkaardige onderzoeken gevoerd? Wat concluderen ze? Wat is het verschil met jouw onderzoek?

%Verwijs bij elke introductie van een term of bewering over het domein naar de vakliteratuur, bijvoorbeeld\autocite{Hykes2013}! Denk zeker goed na welke werken je refereert en waarom.

%Draag zorg voor correcte literatuurverwijzingen! Een bronvermelding hoort thuis \emph{binnen} de zin waar je je op die bron baseert, dus niet er buiten! Maak meteen een verwijzing als je gebruik maakt van een bron. Doe dit dus \emph{niet} aan het einde van een lange paragraaf. Baseer nooit teveel aansluitende tekst op eenzelfde bron.

%Als je informatie over bronnen verzamelt in JabRef, zorg er dan voor dat alle nodige info aanwezig is om de bron terug te vinden (zoals uitvoerig besproken in de lessen Research Methods).

% Voor literatuurverwijzingen zijn er twee belangrijke commando's:
% \autocite{KEY} => (Auteur, jaartal) Gebruik dit als de naam van de auteur
%   geen onderdeel is van de zin.
% \textcite{KEY} => Auteur (jaartal)  Gebruik dit als de auteursnaam wel een
%   functie heeft in de zin (bv. ``Uit onderzoek door Doll & Hill (1954) bleek
%   ...'')

%Je mag deze sectie nog verder onderverdelen in subsecties als dit de structuur van de tekst kan verduidelijken.
%PROBLEEMSTELLING
\subsection{Analyse}
Een analyse is essentieel voor het creëren van een kwaliteitsvol softwareontwikkelingsproject. \\Het houdt in dat er samen met de klant, in dit geval Inetum-Realdolmen, wordt overlegd om de behoeften en wensen te identificeren. Dit is een cruciaal onderdeel van het project, omdat het de fundering legt voor de latere ontwikkeling van de applicatie. Recent onderzoek heeft aangetoond dat de implementatie van modelleermethoden, zoals use cases, een intuïtieve manier biedt om deze functionele eisen vast te leggen en te definiëren. Bovendien is gebleken dat het aanvullen van use cases met schermmockups de begrijpelijkheid van functionele vereisten significant kan verbeteren, waardoor de effectiviteit en efficiëntie van de analysefase bijna verdubbeld kunnen worden, zonder een significante toename in de benodigde inspanning \autocite{EffectScreenMockups}. Deze bevindingen benadrukken het belang van visuele hulpmiddelen bij het vergemakkelijken van de communicatie tussen ontwikkelaars en klanten tijdens de analysefase.

\subsection{Locatiegebaseerde technologieën}
Geofencing en iBeacon zijn twee soorten locatiegebaseerde technologieën. Geofencing gebruikt GPS of RFID-technologie om een virtuele geografische grens te creëren, waarmee de app automatisch check-in en check-out tijden kan registreren zodra een werknemer een locatie binnenkomt of verlaat. iBeacon-technologie, gebaseerd op Bluetooth Low Energy (BLE), stelt apparaten in staat om signalen te zenden en te ontvangen binnen een klein bereik, wat verder verfijnt hoe aanwezigheid kan worden gedetecteerd en geregistreerd. Onderzoek door \textcite{Singh2018Geofencing} over locatie gebaseerde services met Geofencing en onderzoek over binnenshuis positioneering met mobile apparaten door \textcite{iBeaconDeployment} heeft de potentie van deze technologieën voor diverse toepassingen, waaronder tijdregistratie, aangetoond.

\subsection{Beveiligde verbinding}
Het veilig verzenden van gevoelige gegevens van de Android app naar een centrale server is een cruciaal aspect van de ontwikkeling van de tijdsregistratie-app. Uit onderzoek over API beveileging door \textcite{APISecurity} benadrukt de noodzaak van encryptie en andere beveiligingsmaatregelen om de integriteit en vertrouwelijkheid van de gegevens te waarborgen.
Om een beveiligde verbinding te creëren tussen de app en de back-end, zal gebruik gemaakt worden van een authenticatie service. Deze service zal de gebruiker toestaan om zich te authenticeren en toegang te krijgen om data te versturen naar de database.

\subsection{Application Programming Interface}
Het is noodzakelijk om communicatie tussen de Android applicatie en de database mogelijk te maken. Hiervoor wordt een Application Programming Interface (API) ontwikkeld. 'Een API biedt een eenvoudige manier om verbinding te maken met softwaresystemen en deze te integreren en uit te breiden.' \autocite{biehl2015api}. De noodzaak van een API in dit onderzoeksproject ontstaat uit de behoefte om de gegevens van de app naar de database te sturen. De API zal de gegevens van de app ontvangen en verwerken, en vervolgens naar de database sturen. De API zal ook de gegevens van de database naar de app sturen. Dit kan bijvoorbeeld gebruikt worden om de gebruiker te voorzien van een overzicht van zijn geregistreerde uren.

\subsection{Database Management System}
Een databasebeheersysteem (DBMS) middleware waarmee programmeurs, databasebeheerders (DBA's), softwaretoepassingen en eindgebruikers gegevens in een database kunnen opslaan, organiseren, openen, opvragen en manipuleren \autocite{Rouse2024}. Een DBMS maakt het niet alleen mogelijk voor opslag van gegevens, zoals de locatie van de gebruiker, de aankomsttijd op deze locatie, en het tijdstip van vertrek, maar biedt ook geavanceerde functionaliteiten voor het beheren van deze data. Dit omvat het aanpassen, verwijderen, of toevoegen van gebruikersinformatie door middel van een gestructureerde en efficiënte benadering.

De keuze van het juiste databasebeheersysteem is essentieel, aangezien het de basis vormt voor het waarborgen van de integriteit, beschikbaarheid en veiligheid van gebruikersgegevens. Volgens een artikel op door \textcite{SilnitskyDatabase} over hoe je een correcte DBMS kiest bij jouw applicatie, moet bij het selecteren van een DBMS rekening gehouden worden met verschillende factoren, zoals schaalbaarheid, betrouwbaarheid en de efficiëntie in dataretentie en -verwerking. Deze criteria zijn bijzonder belangrijk gezien de gevoelige aard van de te beheren data, waaronder locatiegegevens en tijdregistraties, die zowel nauwkeurigheid als privacy vereisen.

%\subsection{Tijdregistratiesystemen}
%Recente ontwikkelingen in tijdregistratiesystemen hebben aangetoond dat de overstap van handmatige naar digitale methoden bedrijven helpt bij het verbeteren van de nauwkeurigheid van werkurenregistratie en het verminderen van fraude. Studies zoals die van \textcite{TimeTrackingTechEvolution} illustreren hoe digitale systemen bijdragen aan betere werkurenmanagement en operationele efficiëntie.

%\subsection{Impact van Nauwkeurige Tijdregistratie}
%Nauwkeurige tijdregistratie is cruciaal voor bedrijfsvoering, vooral in sectoren waar facturatie gebaseerd is op gepresteerde uren. Onderzoek door \textcite{BillingAccuracyImportance} benadrukt hoe precieze tijdregistratie direct invloed heeft op de winstgevendheid en klanttevredenheid. Daarnaast biedt het inzichten in personeelsproductiviteit en projectmanagementefficiëntie.

%\subsection{Open Vragen en Uitdagingen}
%Deze literatuurstudie legt de basis voor het onderzoek naar de ontwikkeling van een tijdsregistratie-app die geofencing en iBeacon-technologie integreert om automatische en nauwkeurige werkurentracking te faciliteren. Door deze technologieën toe te passen, beoogt dit project de efficiëntie van tijdsregistratie te verbeteren en een betrouwbare basis te bieden voor facturatie en projectbeheer binnen organisaties.

%---------- Methodologie ------------------------------------------------------
\section{Methodologie}%
\label{sec:methodologie}

%Hier beschrijf je hoe je van plan bent het onderzoek te voeren. Welke onderzoekstechniek ga je toepassen om elk van je onderzoeksvragen te beantwoorden? Gebruik je hiervoor literatuurstudie, interviews met belanghebbenden (bv.voor requirements-analyse), experimenten, simulaties, vergelijkende studie, risico-analyse, PoC, \ldots?

%Valt je onderwerp onder één van de typische soorten bachelorproeven die besproken zijn in de lessen Research Methods (bv.\ vergelijkende studie of risico-analyse)? Zorg er dan ook voor dat we duidelijk de verschillende stappen terug vinden die we verwachten in dit soort onderzoek!

%Vermijd onderzoekstechnieken die geen objectieve, meetbare resultaten kunnen opleveren. Enquêtes, bijvoorbeeld, zijn voor een bachelorproef informatica meestal \textbf{niet geschikt}. De antwoorden zijn eerder meningen dan feiten en in de praktijk blijkt het ook bijzonder moeilijk om voldoende respondenten te vinden. Studenten die een enquête willen voeren, hebben meestal ook geen goede definitie van de populatie, waardoor ook niet kan aangetoond worden dat eventuele resultaten representatief zijn.

%Uit dit onderdeel moet duidelijk naar voor komen dat je bachelorproef ook technisch voldoen\-de diepgang zal bevatten. Het zou niet kloppen als een bachelorproef informatica ook door bv.\ een student marketing zou kunnen uitgevoerd worden.

%Je beschrijft ook al welke tools (hardware, software, diensten, \ldots) je denkt hiervoor te gebruiken of te ontwikkelen.

%Probeer ook een tijdschatting te maken. Hoe lang zal je met elke fase van je onderzoek bezig zijn en wat zijn de concrete \emph{deliverables} in elke fase?

In deze initiële fase, die twee weken duurt, wordt een grondige behoeftenanalyse uitgevoerd om de specifieke vereisten van de tijdsregistratie-app te identificeren. Dit omvat interviews met belanghebbenden binnen Inetum-Realdolmen. Door deze interacties worden zowel de functionele als de niet-functionele vereisten van de app geïdentificeerd, waarna deze informatie wordt vertaald naar concrete deliverables, inclusief gedetailleerde mockups van de app.

De volgende fase van het project heeft een duur van 3 weken. In deze periode wordt een studie gedaan naar geofencing en iBeacon-technologie. Als ook naar hoe een beveiligde verbinding tussen de database en API kan opgezet worden. Deze fase levert een overzicht van bestaande technologieën en systemen.

De derde fase neemt 6 weken in beslag. Deze fase richt zich op de ontwikkeling van de Android-app en de Java Enterprise Edition back-end. De ontwikkeling omvat ook de integratie van geofencing en iBeacon-technologie. Hierbij is het resultaat de eerste versie van de app en back-end systeem.

In de vierde fase van het project, die 2 weken zal duren, wordt een sales pitch en een demonstratie van de app voorbereid. Dit omvat een presentatie van de functionaliteiten, de gebruiksvriendelijkheid, en de bedrijfswaarde van de app.

Tot slot wordt in de laatste fase van het project, die twee weken duurt, een scriptie opgesteld. Hierin worden het volledige onderzoek en de behaalde resultaten gedetailleerd beschreven. De scriptie vormt een uitgebreid verslag van het uitgevoerde onderzoek, inclusief een beschrijving van de gebruikte methodologie en de uitgevoerde werkzaamheden. Het biedt een overzicht van de genomen stappen tijdens het onderzoek.



%---------- Verwachte resultaten ----------------------------------------------
\section{Verwacht resultaat, conclusie}%
\label{sec:verwachte_resultaten}

%Hier beschrijf je welke resultaten je verwacht. Als je metingen en simulaties uitvoert, kan je hier al mock-ups maken van de grafieken samen met de verwachte conclusies. Benoem zeker al je assen en de onderdelen van de grafiek die je gaat gebruiken. Dit zorgt ervoor dat je concreet weet welk soort data je moet verzamelen en hoe je die moet meten.

%Wat heeft de doelgroep van je onderzoek aan het resultaat? Op welke manier zorgt jouw bachelorproef voor een meerwaarde?

%Hier beschrijf je wat je verwacht uit je onderzoek, met de motivatie waarom. Het is \textbf{niet} erg indien uit je onderzoek andere resultaten en conclusies vloeien dan dat je hier beschrijft: het is dan juist interessant om te onderzoeken waarom jouw hypothesen niet overeenkomen met de resultaten.

De implementatie van dit project zal naar verwachting leiden tot de ontwikkeling van een innovatieve tijdsregistratie-app voor Android, die een significante verbetering biedt ten opzichte van traditionele tijdregistratiemethoden. Door de integratie van locatiegebaseerde technologieën, zal de app automatisch de aanwezigheidsuren van werknemers registreren, wat leidt tot een nauwkeuriger en efficiënter proces voor het vastleggen van werktijden. Dit heeft impact op facturatie, loonadministratie, en projectmanagement binnen organisaties zoals Inetum-Realdolmen.

\printbibliography[heading=bibintoc]

\end{document}