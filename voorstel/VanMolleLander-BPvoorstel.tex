%==============================================================================
% Sjabloon onderzoeksvoorstel bachproef
%==============================================================================
% Gebaseerd op document class `hogent-article'
% zie <https://github.com/HoGentTIN/latex-hogent-article>

\documentclass{hogent-article}

% Invoegen bibliografiebestand
\addbibresource{voorstel.bib}

\studyprogramme{Professionele bachelor toegepaste informatica}
\course{Bachelorproef}
\assignmenttype{Onderzoeksvoorstel}

\academicyear{2023-2024}

\title{Ontwikkeling van een webapplicatie voor het beoordelen van computervaardigheden: een strategie om de digitale kloof op de werkvloer te verkleinen.}

\author{Lander {Van Molle}}
\email{lander.vanmolle@student.hogent.be}


\supervisor[Co-promotor]{J. {Van der Poten} (Inetum-Realdolmen, \href{mailto:jordi.vanderpoten@inetum-realdolmen.world}{jordi.vanderpoten@inetum-realdolmen.world})}

\specialisation{Mobile \& Enterprise development}
\keywords{webapplicatie, digitale kloof, computervaardigheden testen}

\begin{document}

\begin{abstract}
  %Hier schrijf je de samenvatting van je voorstel, als een doorlopende tekst van één paragraaf. Let op: dit is geen inleiding, maar een samenvattende tekst van heel je voorstel met inleiding (voorstelling, kaderen thema), probleemstelling en centrale onderzoeksvraag, onderzoeksdoelstelling (wat zie je als het concrete resultaat van je bachelorproef?), voorgestelde methodologie, verwachte resultaten en meerwaarde van dit onderzoek (wat heeft de doelgroep aan het resultaat?).
  %Enkel de methodologie van de abstract moet in de verleden tijd staan
  In de huidige samenleving zijn computervaardigheden van groot belang. Door de toenemende digitalisering van werkprocessen in het dagelijks leven worden individuen met een lager opleidingsniveau, lager inkomen of hogere leeftijd vaak geconfronteerd met digitale uitsluiting vanwege een gebrek aan basiscomputervaardigheden. Dit leidt tot een aanzienlijke digitale kloof, vooral merkbaar op de werkvloer.
  Om deze uitdaging aan te pakken, stelt dit onderzoek voor om een webapplicatie te ontwikkelen die de digitale vaardigheden van gebruikers op de werkvloer beoordeelt. Naargelang het resultaat van de vaardigheidstesten biedt de werkgever de toekomstige of huidige werknemer de mogelijkheid om ICT-cursussen te volgen. Om dit te bereiken werden de meest cruciale vaardigheden geselecteerd en werden er testen ontwikkeld om de ICT-competenties van de gebruikers te beoordelen.
  De tests werden uitgevoerd binnen een Windows-omgeving waarbij de focus lag op scenario's die de vaardigheden in informatieverwerking, communicatie, bestandsbeheer (zoals het lokaliseren van bestanden op de pc en software-installatie) en tekstverwerking toetsten. Na afloop van de testen kreeg de gebruiker een overzicht van zijn prestaties zodat zijn eventuele tekortkomingen kan verbeteren door het volgen van specifieke ICT-cursussen. De tool is een mogelijke oplossing om de digitale kloof te verkleinen aangezien individuen de mogelijkheid krijgen om zich bij te scholen.
  In de bedrijfssector kunnen werkgevers sollicitanten en personeel testen op eventuele tekortkomingen op het vlak van ICT en hen workshops of cursussen aanbieden. Dit maakt de tool waardevol voor zowel individuen als bedrijven.
\end{abstract}

\tableofcontents

%---------- Inleiding ---------------------------------------------------------

\section{Introductie}%
\label{sec:introductie}

%Waarover zal je bachelorproef gaan? Introduceer het thema en zorg dat volgende zaken zeker duidelijk aanwezig zijn:

%\begin{itemize}
%  \item kaderen thema
%  \item de doelgroep
%  \item de probleemstelling en (centrale) onderzoeksvraag
%  \item de onderzoeksdoelstelling
%\end{itemize}

%Denk er aan: een typische bachelorproef is \textit{toegepast onderzoek}, wat betekent dat je start vanuit een concrete probleemsituatie in bedrijfscontext, een \textbf{casus}. Het is belangrijk om je onderwerp goed af te bakenen: je gaat voor die \textit{ene specifieke probleemsituatie} op zoek naar een goede oplossing, op basis van de huidige kennis in het vakgebied.

%De doelgroep moet ook concreet en duidelijk zijn, dus geen algemene of vaag gedefinieerde groepen zoals \emph{bedrijven}, \emph{developers}, \emph{Vlamingen}, enz. Je richt je in elk geval op it-professionals, een bachelorproef is geen populariserende tekst. Eén specifiek bedrijf (die te maken hebben met een concrete probleemsituatie) is dus beter dan \emph{bedrijven} in het algemeen.

%Formuleer duidelijk de onderzoeksvraag! De begeleiders lezen nog steeds te veel voorstellen waarin we geen onderzoeksvraag terugvinden.

%Schrijf ook iets over de doelstelling. Wat zie je als het concrete eindresultaat van je onderzoek, naast de uitgeschreven scriptie? Is het een proof-of-concept, een rapport met aanbevelingen, \ldots Met welk eindresultaat kan je je bachelorproef als een succes beschouwen?


In onze huidige samenleving maakt men in bijna elke job gebruik van digitale technologieën. Zeker na de COVID-19-pandemie namen computervaardigheden aan belang toe aangezien meer werknemers vanop afstand werkten. ~\autocite{NBERw27422}. Zoals blijkt uit gegevens van de Federale Overheidsdienst Economie, beschikken individuen met een lager opleidingsniveau, lager inkomen of hogere leeftijd beschikken vaak niet over de nodige computervaardigheden waardoor ze digitaal uitgesloten worden. ~\autocite{FederalPublicServiceEconomyKloof}.

De Belgische overheid heeft de ambitie om tegen 2030 een digitaal competente samenleving te realiseren, zoals uiteengezet in de 'Digital Decade' visie~\autocite{DigitalDecade2030}. Dit onderzoeksvoorstel focust zicht op de digitalisering op de werkvloer. Het is de bedoeling om sollicitanten of huidige werknemers met een beperkte ICT-kennis (bijvoorbeeld arbeiders, magazijniers, logistiek medewerkers) te testen op hun computervaardigheden zodat zij zich, indien gewenst, kunnen bijscholen. % focus op oudere werknemers en werknemers in KMO's in de gezondheidszorgsector

Centraal in dit onderzoek staat de vraag hoe de ontwikkeling van een webapplicatie de computervaardigheden van gebruikers op de werkvloer effectief kan beoordelen bijdragen aan het verminderen van digitale uitsluiting? Om deze digitale kloof te overbruggen is er nood aan een webtool om de computervaardigheden van personen te kunnen beoordelen. %Dit onderzoeksvoorstel analyseert het ontwikkelen van een webtool om computervaardigheden op de werkvloer te testen. 
Deze tool geeft aan welke computervaardigheden sollicitanten best verbeteren en helpt werkgevers bij het beoordelen van sollicitanten. Het doel van de webtool is de digitale kloof op de werkvloer te verkleinen.

Het resultaat van dit onderzoek is een Proof-of-Concept van een webapplicatie, waarbij de gebruiker een score ontvangt gebaseerd op correctheid, tijd en het aantal benodigde hints bij elke vraag. Deze scores geven inzicht in de huidige digitale vaardigheden en dienen bovendien ook als leidraad voor gerichte training en ontwikkeling, met als doel het verkleinen van de digitale kloof.


%---------- Stand van zaken ---------------------------------------------------

\section{Literatuurstudie}%
\label{sec:state-of-the-art}

%Hier beschrijf je de \emph{state-of-the-art} rondom je gekozen onderzoeksdomein, d.w.z.\ een inleidende, doorlopende tekst over het onderzoeksdomein van je bachelorproef. Je steunt daarbij heel sterk op de professionele \emph{vakliteratuur}, en niet zozeer op populariserende teksten voor een breed publiek. Wat is de huidige stand van zaken in dit domein, en wat zijn nog eventuele open vragen (die misschien de aanleiding waren tot je onderzoeksvraag!)?

%Je mag de titel van deze sectie ook aanpassen (literatuurstudie, stand van zaken, enz.). Zijn er al gelijkaardige onderzoeken gevoerd? Wat concluderen ze? Wat is het verschil met jouw onderzoek?

%Verwijs bij elke introductie van een term of bewering over het domein naar de vakliteratuur, bijvoorbeeld~\autocite{Hykes2013}! Denk zeker goed na welke werken je refereert en waarom.

%Draag zorg voor correcte literatuurverwijzingen! Een bronvermelding hoort thuis \emph{binnen} de zin waar je je op die bron baseert, dus niet er buiten! Maak meteen een verwijzing als je gebruik maakt van een bron. Doe dit dus \emph{niet} aan het einde van een lange paragraaf. Baseer nooit teveel aansluitende tekst op eenzelfde bron.

%Als je informatie over bronnen verzamelt in JabRef, zorg er dan voor dat alle nodige info aanwezig is om de bron terug te vinden (zoals uitvoerig besproken in de lessen Research Methods).

% Voor literatuurverwijzingen zijn er twee belangrijke commando's:
% \autocite{KEY} => (Auteur, jaartal) Gebruik dit als de naam van de auteur
%   geen onderdeel is van de zin.
% \textcite{KEY} => Auteur (jaartal)  Gebruik dit als de auteursnaam wel een
%   functie heeft in de zin (bv. ``Uit onderzoek door Doll & Hill (1954) bleek
%   ...'')

%Je mag deze sectie nog verder onderverdelen in subsecties als dit de structuur van de tekst kan verduidelijken.
%PROBLEEMSTELLING
Volgens een onderzoek van de ~\textcite{VlaamseVeerkracht} mist 46\% van de Vlaamse bevolking van 16 tot 74 jaar de nodige digitale basisvaardigheden: 37\% heeft lage digitale vaardigheden en 8\% heeft geen digitale vaardigheden of heeft geen internet gebruikt in de voorbije 3 maanden.

~\textcite{DigitaleInclusieBarometer} bevestigt in een onderzoek van de Koning Boudewijnstichting dat een groot deel van de Belgen kwetsbaar zijn voor uitsluiting van essentiële diensten zoals werk, onderwijs en gezondheidszorg omdat zij over onvoldoende digitale vaardigheden beschikken.

De digitale ongelijkheid is een realiteit. ~\textcite{IedereenDigitaleKloof} van het Vlaams Steunpunt Nieuwe Geletterdheid merkt in een artikel op dat de digitale kloof geen ééndimensioneel probleem is. \indent Naast het bezit van computers en de instrumentele vaardigheden bestaat de digitale kloof immers uit nog andere facetten. Zo heeft men tevens informatievaardigheden nodig. Dit zijn methodes om online informatie op te sporen: "zoeken, selecteren, begrijpen, evalueren en verwerken". Maar zelfs tot het toepassen van digitale informatievaardigheden op een eenvoudig niveau, namelijk het gebruiken van een zoekmachine, is 1/3 van de Belgische bevolking niet in staat.

De coronapandemie is dé gangmaker geweest voor de digitalisering van de maatschappij. Volgens ~\textcite{Unia} ging het digitaliseringsproces de afgelopen jaren gepaard met het geleidelijk verdwijnen van een aantal fysieke loketten. Daar konden nochtans vele burgers terecht en werden ze geholpen als ze administratieve problemen hadden. Deze toegenomen digitalisering heeft gevolgen op veel maatschappelijke domeinen zoals bankdiensten, mobiliteit, gezondheidszorg en onderwijs.

%VERTELD IN DE INTRODUCTIE: 
Dit onderzoeksvoorstel richt zich op de digitalisering op de werkvloer, met speciale aandacht voor sollicitanten of huidige werknemers met beperkte ICT-kennis (zoals arbeiders, magazijniers, logistiek medewerkers). Het doel is om hun computervaardigheden te testen en hen de mogelijkheid te bieden zich bij te scholen, in lijn met het doel van de Vlaamse overheid om minstens 80\% van de bevolking digitaal vaardig te maken.
Binnen dit onderzoek ligt de focus op het testen van essentiële digitale vaardigheden die gedefinieerd worden in het DigComp-raamwerk van het Joint Research Centre van de Europese Commissie. Deze omvatten het opzoeken en verwerken van informatie, het maken van PDF-documenten, het vinden van bestanden op de computer, e-mailen en het uploaden van bestanden. Deze vaardigheden zijn cruciaal voor een effectieve digitale communicatie en organisatie in de moderne werkomgeving (Bron: DigComp Framework).
Om de computervaardigheden van individuen op een betrouwbare en nauwkeurige manier te beoordelen, is het ontwikkelen van een webapplicatie met concrete ICT-opdrachten gepland. Traditionele vragenlijsten, zoals die op de website van VDAB, bieden vooral theoretische vragen en geven geen nauwkeurig beeld van praktische vaardigheden.
De geplande webapplicatie zal in tegenstelling tot deze traditionele methoden praktijkgerichte oefeningen bevatten. Dit is in overeenstemming met bevindingen uit onderzoek dat aangeeft dat het onderhouden van georganiseerde informatierepositories bijdraagt aan een groter vertrouwen in eigen kennis (zie: https://onlinelibrary.wiley.com/doi/full/10.1002/acp.3277).
Daarnaast onderstrepen de DESI-rapporten 2021 en 2022 de noodzaak van dit onderzoek. Het DESI 2022 rapport benadrukt dat, hoewel België goede stappen heeft gemaakt in digitale connectiviteit en de integratie van digitale technologieën in bedrijven, er aanzienlijke ruimte is voor verbetering in digitale vaardigheden in het onderwijs en de bredere beroepsbevolking. Deze contextuele informatie ondersteunt de relevantie van het onderzoeksvoorstel binnen het bredere maatschappelijke en economische kader van België.
Tot slot biedt het DigComp-raamwerk van het Joint Research Centre van de Europese Commissie een nuttig referentiekader voor het definiëren van digitale competenties en vaardigheden, wat relevant kan zijn voor de ontwikkeling van de webapplicatie %(zie: https://joint-research-centre.ec.europa.eu/digcomp/digcomp-framework_en).

%SOFTWARE
%uitleggen website
Bovenvermelde informatie rechtvaardigt de keuze voor een webapplicatie. Immers, gebruikers die over onvoldoende basis ICT-kennis beschikken, slagen er niet in om software correct te installeren. Het gebruik van een webapplicatie maakt het mogelijk om de testen in een veilige virtuele omgeving uit te voeren zodat de gebruiker geen schade kan aanbrengen aan zijn eigen digitale omgeving. De webapplicatie zal dus enkel dienen als communicatie middel tussen de de Windows-omgeving, waar de testen zullen worden uitegvoerd en de gebruiker.

%uitleggen OS
De keuze voor een Windows-omgeving om computervaardigheidstesten uit te voeren, is ingegeven door de brede verspreiding en populariteit van Windows als besturingssysteem. Volgens een rapport van StatCounter, heeft Windows een aanzienlijk marktaandeel in desktopbesturingssystemen in België, wat de algemene prevalentie en acceptatie ervan in de professionele wereld aantoont~\autocite{StatCounterOSMarketShare}. Dit wordt verder ondersteund door het feit dat Windows vaak het standaard besturingssysteem is bij de aanschaf van nieuwe computers~\autocite{ProfolusWindowsPopularity}.

%uitleggen VM
Het opzetten van deze Windows-omgevingen kan gebeuren door middel van het gebruiken van een virtualisatiesoftwarepakket.

%uitleggen connectie VM met website
Er is nood aan een manier om te kunnen interacten met de webapplicatie en een windows omgeving. Om deze windows omgeving te tonen in een website is er een tool nodig om een verbinging te maken met deze windows omgeving. Apache Guacamole is het meest geschikte framework dat ondersteuning biedt voor een verscheidenheid aan protocollen waaronder VNC, RDP, SSH en meer. Hierdoor kun je verschillende soorten externe verbindingen beheren en schakelen tussen diverse systemen zonder de noodzaak van aparte tools. Dit framework biedt geavanceerde beveiligingsfuncties zoals o.a. end-to-end SSL-codering, toegangscontrole op basis van gebruikers en IP-adressen. Apache Guacamole kan eveneens worden geïmplementeerd als een Docker-container, wat de installatie en configuratie vereenvoudigt. Dit vergemakkelijkt het aanpassen van Apache Guacamole aan verschillende omgevingen of het nu on-premises, in de cloud of in een hybride setup is. ~\autocite{ApacheGuacamole} % Nieuw: https://www.ronpub.com/OJCC_2023v8i1n01_Baun.pdf

NoVNC is een minder gebruiksvriendelijk framework. Gebruikers dienen bij NoVNC handmatig VNC-parameters in te voeren en instellingen te configureren, hetgeen technische kennis vereist. Het aanpassen van de gebruikersinterface en functionaliteit van noVNC is technisch uitdagender dan bij Guacamole. ~\autocite{noVNC} %https://novnc.com/noVNC/ 
Rekening houdend met de doelgroep is het aangewezen om een Apache Guacamole als framework te gebruiken.

%uitleggen testen
Nu dat er een connectie kan gemaakt worden met de windows-omeginvg om de testen uit te voeren moeten nog de testen ontwikkeld. Op de virtuele windows omgevingen moet er voorg een applicatie geinstalleerd worden om de gebruiker de testen uit te leggen en bovendien data kan bijhouden hoe de testen worden uitgevoerd, hoe lang de testen duren en hoeveel hints er nodig zijn.  Er kan keuze gemaakt worden uit Java of C\# om deze applicatie te ontwikkelen.


%Om data uit de virtuele Windows-omgeving te koppelen aan een score, is er nood aan een vooraf geïnstalleerde softwareapplicatie. Er kan keuze gemaakt worden uit Java of C\#. De score van elke test is bepaald door het aantal correct uitgevoerde opdrachten en de tijd die nodig is om de test uit te voeren. Deze score kan opgeslagen worden in CSV-formaat.
%C# is de beste programmeertaal om de testen te maken voor de windows omgeving.
%C# maakt gebruik van het .NET Framework (of .NET Core voor cross-platform ontwikkeling) dat uitgebreide ondersteuning biedt voor Windows-ontwikkeling. Het biedt een breed scala aan bibliotheken en hulpmiddelen voor het bouwen van Windows-toepassingen met een rijke gebruikersinterface, services en communicatie.


%uitleggen score
%Om de testresultaten te visualiseren wordt gebruik gemaakt van Pandas. ~\textcite{NVIDIA} besluit: "[Deze visualisatie software beschikt over] robuuste I/O-tools voor het laden van gegevens uit platte bestanden (CSV en delimited), Excel-bestanden, databases en het opslaan/laden van gegevens uit het supersnelle HDF5-formaat" %%https://www.nvidia.com/en-us/glossary/data-science/pandas-python/

Dit geautomatiseerde proces bespaart tijd voor de werkgever en de sollicitant.

Na het verkrijgen van de testresultaten kan de werkgever de sollicitant of huidige werknemer cursussen aanbieden om zijn computervaardigheden bij te schaven. Op die manier zijn werknemers up-to-date met bedrijfsspecfieke software en beschikken ze over basis ICT-kennis. Dit kan een eerste stap zijn naar het verkleinen van de digitale kloof.


%---------- Methodologie ------------------------------------------------------
\section{Methodologie}%
\label{sec:methodologie}

Hier beschrijf je hoe je van plan bent het onderzoek te voeren. Welke onderzoekstechniek ga je toepassen om elk van je onderzoeksvragen te beantwoorden? Gebruik je hiervoor literatuurstudie, interviews met belanghebbenden (bv.~voor requirements-analyse), experimenten, simulaties, vergelijkende studie, risico-analyse, PoC, \ldots?

Valt je onderwerp onder één van de typische soorten bachelorproeven die besproken zijn in de lessen Research Methods (bv.\ vergelijkende studie of risico-analyse)? Zorg er dan ook voor dat we duidelijk de verschillende stappen terug vinden die we verwachten in dit soort onderzoek!

Vermijd onderzoekstechnieken die geen objectieve, meetbare resultaten kunnen opleveren. Enquêtes, bijvoorbeeld, zijn voor een bachelorproef informatica meestal \textbf{niet geschikt}. De antwoorden zijn eerder meningen dan feiten en in de praktijk blijkt het ook bijzonder moeilijk om voldoende respondenten te vinden. Studenten die een enquête willen voeren, hebben meestal ook geen goede definitie van de populatie, waardoor ook niet kan aangetoond worden dat eventuele resultaten representatief zijn.

Uit dit onderdeel moet duidelijk naar voor komen dat je bachelorproef ook technisch voldoen\-de diepgang zal bevatten. Het zou niet kloppen als een bachelorproef informatica ook door bv.\ een student marketing zou kunnen uitgevoerd worden.

Je beschrijft ook al welke tools (hardware, software, diensten, \ldots) je denkt hiervoor te gebruiken of te ontwikkelen.

Probeer ook een tijdschatting te maken. Hoe lang zal je met elke fase van je onderzoek bezig zijn en wat zijn de concrete \emph{deliverables} in elke fase?

%---------- Verwachte resultaten ----------------------------------------------
\section{Verwacht resultaat, conclusie}%
\label{sec:verwachte_resultaten}

Hier beschrijf je welke resultaten je verwacht. Als je metingen en simulaties uitvoert, kan je hier al mock-ups maken van de grafieken samen met de verwachte conclusies. Benoem zeker al je assen en de onderdelen van de grafiek die je gaat gebruiken. Dit zorgt ervoor dat je concreet weet welk soort data je moet verzamelen en hoe je die moet meten.

Wat heeft de doelgroep van je onderzoek aan het resultaat? Op welke manier zorgt jouw bachelorproef voor een meerwaarde?

Hier beschrijf je wat je verwacht uit je onderzoek, met de motivatie waarom. Het is \textbf{niet} erg indien uit je onderzoek andere resultaten en conclusies vloeien dan dat je hier beschrijft: het is dan juist interessant om te onderzoeken waarom jouw hypothesen niet overeenkomen met de resultaten.



\printbibliography[heading=bibintoc]

\end{document}